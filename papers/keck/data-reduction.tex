\documentclass[useAMS,usenatbib]{mn2e}
\usepackage{wjh-mnras-custom}

\newcommand\wav[1]{\ensuremath{\lambda #1}}
\newcommand\wavwav[1]{\ensuremath{\lambda\!\lambda #1}}

\graphicspath{
  {../Stamps/},
  }

\newcommand\OnePV[3]{%
  \includegraphics[width=#2\linewidth]
  {p84-#1-stamp-#3-stages}
}

\newcommand\TwoPV[4]{%
  \begin{tabular}{@{}l@{}}
    (a)\\
    \includegraphics[width=#3\linewidth]
    {p84-#1-stamp-#4-stages}\\
    \\
    (b)\\
    \includegraphics[width=#3\linewidth]
    {p84-#2-stamp-#4-stages}
  \end{tabular}
}

\title{Description of Keck HIRES data reduction steps}
\author{William J. Henney}

\begin{document}
\maketitle
\begin{abstract}
  Material that may get included in a paper.  
  Describes the data reduction steps carried out for the 
  Keck HIRES spectra of the Orion proplyds.
  Basic reduction steps: 
  order extraction, wavelength calibration, detilting,
  de-overlapping, flat-fielding.  
  Decomposition into continuum plus line.  
  Deconvolution of fine-structure multiplets. 
  Decomposition of line into sky plus nebula plus proplyd. 
  Sample results for slit p84 
  from the giant proplyd 244-440. 
\end{abstract}

\section{Basic Reduction}
\label{sec:basic}



\section{Decomposition}
\label{sec:decomp}

\subsection{Continuum subtraction}
\label{sec:cont}

We fit the continuum and subtract it. 

\subsection{Deconvolution of multiplets}
\label{sec:deconv}


For the \ion{O}{1} lines 

\subsection{Nebular subtraction}
\label{sec:neb}



\section{Sample results: 244-440 slit 84}
\label{sec:sample}


\subsection{High-ionization lines}
\label{sec:high}


\begin{figure*}
  \centering
  \TwoPV{O_III_5007}{C_II_6578}{0.75}{line}
  \caption{(a) Collisionally excited forbidden line of doubly ionized oxygen: [\ion{O}{3}] \wav{5007}.  (b)~Recombination line of singly ionized carbon: \ion{C}{2} \wav{6578}}
  \label{fig:p84-oiii-cii-lines}
\end{figure*}

\begin{figure*}
  \centering
  \TwoPV{He_I_T_5876}{He_I_S_6678}{0.75}{line}
  \caption{Recombination lines of neutral helium: (a) \ion{He}{1} \wav{5876} triplet;  (b)~\ion{He}{1} \wav{6678} singlet.}
  \label{fig:p84-oi-collisional-lines}
\end{figure*}


\subsection{Moderate-ionization lines}
\label{sec:moderate}


\begin{figure*}
  \centering
  \OnePV{S_III_6312}{0.75}{line}
  \caption{Collisionally excited line of doubly ionized sulfur: [\ion{S}{3}] \wav{6312}.}
  \label{fig:p84-siii-line}
\end{figure*}

\begin{figure*}
  \centering
  \TwoPV{Cl_III_5518}{Cl_III_5538}{0.75}{line}
  \caption{Collisionally excited lines of doubly ionized chlorine: (a)~[\ion{Cl}{3}] \wav{5518};  (b)~[\ion{Cl}{3}] \wav{5538}.}
  \label{fig:p84-cl-iii-lines}
\end{figure*}



\subsection{Low-ionization lines}
\label{sec:low}

\begin{figure*}
  \centering
  \TwoPV{N_II_5755}{N_II_6548}{0.75}{line}
  \caption{Collisionally excited lines of singly ionized nitrogen: (a)~[\ion{N}{2}] \wav{5575} auroral line;  (b)~[\ion{N}{2}] \wav{6548} nebular line.}
  \label{fig:p84-nii-lines}
\end{figure*}

\begin{figure*}
  \centering
  \TwoPV{S_II_6716}{S_II_6731}{0.75}{line}
  \caption{Collisionally excited lines of singly ionized sulfur: (a)~[\ion{S}{2}] \wav{6731};  (b)~[\ion{S}{2}] \wav{6716}.}
  \label{fig:p84-sii-lines}
\end{figure*}

\begin{figure*}
  \centering
  \TwoPV{Fe_III_4881}{Fe_III_5270}{0.75}{line}
  \caption{Collisionally excited lines of doubly ionized iron: (a)~[\ion{Fe}{3}] \wav{4881};  (b)~[\ion{Fe}{3}] \wav{5270}.}
  \label{fig:p84-fe-iii-lines}
\end{figure*}


\subsection{Neutral collisional lines}
\label{sec:neutral}

\subsection{[\ion{O}{1}] lines}
\label{sec:oi-forbidden}

\begin{figure*}
  \centering
  \TwoPV{O_I_6300}{O_I_5577}{0.75}{line}
  \caption{Collisionally excited forbidden lines of neutral oxygen: [\ion{O}{1}] \wav{6300} and \wav{5577}}
  \label{fig:p84-oi-collisional-lines}
\end{figure*}


\subsection{Fluorescent lines}
\label{sec:fluor}

\subsubsection{[\ion{N}{1}] lines}
\label{sec:ni}

\begin{figure*}
  \centering
  \TwoPV{N_I_5198}{N_I_5200}{0.75}{line}
  \caption{Continuum fluorescence-excited forbidden lines of neutral nitrogen: [\ion{N}{1}] \wavwav{5198,5200}}
  \label{fig:p84-ni-lines}
\end{figure*}

\subsubsection{\ion{O}{1} lines}
\label{sec:oi-permitted}

\begin{figure*}
  \centering
  \TwoPV{O_I_6046}{O_I_7002}{1.0}{doublet}
  \caption{Continuum fluorescence-excited forbidden lines of neutral oxygen: [\ion{O}{1}] \wav{6046} and \wav{7002}.}
  \label{fig:oi-permitted-lines}
\end{figure*}

\subsubsection{\ion{Fe}{2} lines}
\label{sec:feii}



\begin{figure*}
  \centering
  \TwoPV{Fe_II_5159}{Fe_II_5262}{0.75}{line}
  \caption{Continuum fluorescence-excited forbidden lines of singly-ionized iron: [\ion{Fe}{2}] \wav{5159} and \wav{5262}.}
  \label{fig:fe-ii-lines}
\end{figure*}

\subsubsection{\ion{Si}{2} lines}
\label{sec:silicon}

\begin{figure*}
  \centering
  \TwoPV{Si_II_6347}{Si_II_6371}{0.75}{line}
  \caption{Continuum fluorescence/recombination-excited permitted lines of singly-ionized silicon: \ion{Si}{2} \wav{6347} and \wav{6371}.}
  \label{fig:si-ii-lines}
\end{figure*}




\end{document}

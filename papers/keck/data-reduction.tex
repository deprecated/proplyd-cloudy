\documentclass[useAMS,usenatbib]{mn2e}
\usepackage{wjh-mnras-custom}
\usepackage{savesym}
\savesymbol{iint}
\savesymbol{iiint}
\savesymbol{iiiint}
\savesymbol{idotsint}
\usepackage{mhchem}

\newcommand\wav[1]{\ensuremath{\lambda #1}}
\newcommand\wavwav[1]{\ensuremath{\lambda\!\lambda #1}}

\graphicspath{
  {figs/},
  }

\newcommand\OnePV[3]{%
  \includegraphics[width=#2\linewidth]
  {p84-#1-stamp-#3-stages}
}

\newcommand\OnePVV[3]{%
  \begin{tabular}{@{}l@{}}
    (a)\\
    \includegraphics[width=#2\linewidth]
    {p84-#1-stamp-#3-stages}\\
    \\
    (b)\\
    \includegraphics[width=#2\linewidth]
    {p85-#1-stamp-#3-stages}
  \end{tabular}
}

\newcommand\TwoPV[4]{%
  \begin{tabular}{@{}l@{}}
    (a)\\
    \includegraphics[width=#3\linewidth]
    {p84-#1-stamp-#4-stages}\\
    \\
    (b)\\
    \includegraphics[width=#3\linewidth]
    {p84-#2-stamp-#4-stages}
  \end{tabular}
}

\newcommand\TwoPVb[4]{%
  \begin{tabular}{@{}l@{}}
    (a)\\
    \includegraphics[width=#3\linewidth]
    {p85-#1-stamp-#4-stages}\\
    \\
    (b)\\
    \includegraphics[width=#3\linewidth]
    {p85-#2-stamp-#4-stages}
  \end{tabular}
}

\title{Description of Keck HIRES data reduction steps}
\author{William J. Henney}

\begin{document}
\maketitle
\begin{abstract}
  Material that may get included in a paper.  
  Describes the data reduction steps carried out for the 
  Keck HIRES spectra of the Orion proplyds.
  Basic reduction steps: 
  order extraction, wavelength calibration, detilting,
  de-overlapping, flat-fielding.  
  Decomposition into continuum plus line.  
  Deconvolution of fine-structure multiplets. 
  Decomposition of line into sky plus nebula plus proplyd. 
  Sample results for slit p84 
  from the giant proplyd 244-440. 
\end{abstract}

\section{Basic Reduction}
\label{sec:basic}

% Let's try a subscript: \(X_{\ion{O}{3}}\)

% {\tiny Let's do something small: \(\ion{Fe}{3}\) or \scIon{Fe}{iii}}

% {
% \itshape
% XXXX [\ion{Fe}{2}] YYY
% }

With \texttt{mhchem} package: 
\ce{O^2+ 2s^2 2p^2 ^1D_2 -> 2s^2 2p^2 ^3P_{0,1,2} }

With normal math commands:
\(
\mathrm{O^{2+}\ 2s^2\,2p^2\,{}^1D_2 \to 2s^2\,2p^2\,{}^3P_{0,1,2}}
\)

\section{Decomposition}
\label{sec:decomp}

\subsection{Continuum subtraction}
\label{sec:cont}

We fit the continuum and subtract it. 

\subsection{Deconvolution of multiplets}
\label{sec:deconv}


For the \ion{O}{1} lines 

\subsection{Nebular subtraction}
\label{sec:neb}



\section{Sample results: 244-440 slit 84}
\label{sec:sample}


\subsection{High-ionization lines}
\label{sec:high}

\subsubsection{[\ion{O}{3}] forbidden lines}
\begin{figure*}
  \centering
  \TwoPV{O_III_5007}{C_II_6578}{0.75}{line}
  \caption{(a) Collisionally excited forbidden line of doubly ionized oxygen: [\ion{O}{3}] \wav{5007}.  (b)~Recombination line of singly ionized carbon: \ion{C}{2} \wav{6578}}
  \label{fig:p84-oiii-cii-lines}
\end{figure*}
\begin{figure*}
  \centering
  \TwoPVb{O_III_5007}{C_II_6578}{0.75}{line}
  \caption{(a) Collisionally excited forbidden line of doubly ionized oxygen: [\ion{O}{3}] \wav{5007}.  (b)~Recombination line of singly ionized carbon: \ion{C}{2} \wav{6578}}
  \label{fig:p84-oiii-cii-lines}
\end{figure*}

The only [\ion{O}{3}] lines that fall in the wavelength range of our observations
are the three components of the \(^1\)D--\(^3\)P 5007, 4959, and 4931 multiplet is the only one that is detected,
with theoretical relative intensities 1:0.34:0.00013.     
Only the 5007 component is shown in Figure~\ref{fig:p84-oiii-cii-lines}.  
The 4959 component shows identical structure, 
while the very weak 4931 component is barely detected.   

The line is very strong from the nebula (X times H b) 
but significantly weaker from the proplyd (Y times Hb).   
As a result the proplyd contrast against the underlying 
nebula line is only 10--20\%.   
Nonetheless, the very high S/N allows a reliable separation 
of the proplyd and nebula contributions, 
as shown in the right column of Figure~\ref{fig:p84-oiii-cii-lines}.   
The proplyd emission separates into blue and red components 
peaking at \(V_{\mathrm{prop}} = -19\) and \(+15\) km/s 
in the proplyd reference frame.   
The red component is about 3 times weaker and more compact 
and shows a larger mean spatial shift of 2 arcsec from the proplyd center, 
as opposed to 1 arcsec for the blue component.  

\subsubsection{\ion{C}{2} permitted line}

This line is produced from the recombination of C\(^{2+}\).   

It is 200 times weaker than [\ion{O}{3}] 5007
and is a similar intensity to the adjacent continuum.
However, the proplyd shows a contrast of about 100\% with respect to the nebula,
which again allows for a reliable separation.
The proplyd emission is extremely similar to that of  [\ion{O}{3}]. 

It is an important question whether the apparent complete lack of emission
around the proplyd systemic velocity is real or whether it is an artifact
induced by extinction of background nebular emission by the proplyd.    \citet{1999AJ....118.2350H} discuss this issue in detail
and show that an effective proplyd extinction as low as \(A_V = 0.2\)
would be enough to fill-in this missing emission.
However, there are two reasons to believe that this does not occur in 244-440.
First, the two-dimensional line profile would appear very unnatural,
since the extra emission would appear at spatial positions
in the core of the proplyd, rather than displaced along the slit,
where the peak of the high ionization emission occurs.
Secondly, the same amount of proplyd extinction
would have a much smaller effect on C II than it would on [\ion{O}{3}] 


The conclusion is that in 244-440 is not affected
by internal extinction of the nebular emission.
This implies that either the extended neutral core of the proplyd
is optically thin at visual wavelengths or that the majority of the nebular emission
arises from gas that is foreground to the proplyd.
This is consistent with the kinematic profile of this proplyd,
which suggests that it is seen head-on
and therefore lies behind its ionizing source.   


\begin{figure*}
  \centering
  \TwoPV{He_I_T_5876}{He_I_S_6678}{0.75}{line}
  \caption{Recombination lines of neutral helium: (a) \ion{He}{1} \wav{5876} triplet;  (b)~\ion{He}{1} \wav{6678} singlet.}
  \label{fig:p84-oi-collisional-lines}
\end{figure*}
\begin{figure*}
  \centering
  \TwoPVb{He_I_T_5876}{He_I_S_6678}{0.75}{line}
  \caption{Recombination lines of neutral helium: (a) \ion{He}{1} \wav{5876} triplet;  (b)~\ion{He}{1} \wav{6678} singlet.}
  \label{fig:p84-oi-collisional-lines}
\end{figure*}


\subsection{Moderate-ionization lines}
\label{sec:moderate}


\begin{figure*}
  \centering
  \OnePVV{S_III_6312}{0.75}{line}
  \caption{Collisionally excited line of doubly ionized sulfur: [\ion{S}{3}] \wav{6312}.}
  \label{fig:p84-siii-line}
\end{figure*}

\begin{figure*}
  \centering
  \TwoPV{Cl_III_5518}{Cl_III_5538}{0.75}{line}
  \caption{Collisionally excited lines of doubly ionized chlorine: (a)~[\ion{Cl}{3}] \wav{5518};  (b)~[\ion{Cl}{3}] \wav{5538}.}
  \label{fig:p84-cl-iii-lines}
\end{figure*}
\begin{figure*}
  \centering
  \TwoPVb{Cl_III_5518}{Cl_III_5538}{0.75}{line}
  \caption{Collisionally excited lines of doubly ionized chlorine: (a)~[\ion{Cl}{3}] \wav{5518};  (b)~[\ion{Cl}{3}] \wav{5538}.}
  \label{fig:p84-cl-iii-lines}
\end{figure*}

\clearpage

\subsection{Low-ionization lines}
\label{sec:low}

\begin{figure*}
  \centering
  \TwoPV{N_II_5755}{N_II_6548}{0.75}{line}
  \caption{Collisionally excited lines of singly ionized nitrogen: (a)~[\ion{N}{2}] \wav{5575} auroral line;  (b)~[\ion{N}{2}] \wav{6548} nebular line.}
  \label{fig:p84-nii-lines}
\end{figure*}
\begin{figure*}
  \centering
  \TwoPVb{N_II_5755}{N_II_6548}{0.75}{line}
  \caption{Collisionally excited lines of singly ionized nitrogen: (a)~[\ion{N}{2}] \wav{5575} auroral line;  (b)~[\ion{N}{2}] \wav{6548} nebular line.}
  \label{fig:p84-nii-lines}
\end{figure*}

\begin{figure*}
  \centering
  \TwoPV{S_II_6716}{S_II_6731}{0.75}{line}
  \caption{Collisionally excited lines of singly ionized sulfur: (a)~[\ion{S}{2}] \wav{6731};  (b)~[\ion{S}{2}] \wav{6716}.}
  \label{fig:p84-sii-lines}
\end{figure*}
\begin{figure*}
  \centering
  \TwoPVb{S_II_6716}{S_II_6731}{0.75}{line}
  \caption{Collisionally excited lines of singly ionized sulfur: (a)~[\ion{S}{2}] \wav{6731};  (b)~[\ion{S}{2}] \wav{6716}.}
  \label{fig:p84-sii-lines}
\end{figure*}

\begin{figure*}
  \centering
  \TwoPV{Fe_III_4881}{Fe_III_5270}{0.75}{line}
  \caption{Collisionally excited lines of doubly ionized iron: (a)~[\ion{Fe}{3}] \wav{4881};  (b)~[\ion{Fe}{3}] \wav{5270}.}
  \label{fig:p84-fe-iii-lines}
\end{figure*}
\begin{figure*}
  \centering
  \TwoPVb{Fe_III_4881}{Fe_III_5270}{0.75}{line}
  \caption{Collisionally excited lines of doubly ionized iron: (a)~[\ion{Fe}{3}] \wav{4881};  (b)~[\ion{Fe}{3}] \wav{5270}.}
  \label{fig:p84-fe-iii-lines}
\end{figure*}


\clearpage
\subsection{Neutral collisional lines}
\label{sec:neutral}

\clearpage
\subsection{[\ion{O}{1}] lines}
\label{sec:oi-forbidden}

\begin{figure*}
  \centering
  \TwoPV{O_I_6300}{O_I_5577}{0.75}{line}
  \caption{Collisionally excited forbidden lines of neutral oxygen: [\ion{O}{1}] \wav{6300} and \wav{5577}}
  \label{fig:p84-oi-collisional-lines}
\end{figure*}
\begin{figure*}
  \centering
  \TwoPVb{O_I_6300}{O_I_5577}{0.75}{line}
  \caption{Collisionally excited forbidden lines of neutral oxygen: [\ion{O}{1}] \wav{6300} and \wav{5577}}
  \label{fig:p84-oi-collisional-lines}
\end{figure*}


\clearpage
\subsection{Fluorescent lines}
\label{sec:fluor}

\subsubsection{[\ion{N}{1}] lines}
\label{sec:ni}

\begin{figure*}
  \centering
  \TwoPV{N_I_5198}{N_I_5200}{0.75}{line}
  \caption{Continuum fluorescence-excited forbidden lines of neutral nitrogen: [\ion{N}{1}] \wavwav{5198,5200}}
  \label{fig:p84-ni-lines}
\end{figure*}
\begin{figure*}
  \centering
  \TwoPVb{N_I_5198}{N_I_5200}{0.75}{line}
  \caption{Continuum fluorescence-excited forbidden lines of neutral nitrogen: [\ion{N}{1}] \wavwav{5198,5200}}
  \label{fig:p84-ni-lines}
\end{figure*}

\subsubsection{\ion{O}{1} lines}
\label{sec:oi-permitted}

\begin{figure*}
  \centering
  \TwoPV{O_I_6046}{O_I_7002}{1.0}{doublet}
  \caption{Continuum fluorescence-excited forbidden lines of neutral oxygen: [\ion{O}{1}] \wav{6046} and \wav{7002}.}
  \label{fig:oi-permitted-lines}
\end{figure*}
\begin{figure*}
  \centering
  \TwoPVb{O_I_6046}{O_I_7002}{1.0}{doublet}
  \caption{Continuum fluorescence-excited forbidden lines of neutral oxygen: [\ion{O}{1}] \wav{6046} and \wav{7002}.}
  \label{fig:oi-permitted-lines}
\end{figure*}

\subsubsection{\ion{Fe}{2} lines}
\label{sec:feii}



\begin{figure*}
  \centering
  \TwoPV{Fe_II_5159}{Fe_II_5262}{0.75}{line}
  \caption{Continuum fluorescence-excited forbidden lines of singly-ionized iron: [\ion{Fe}{2}] \wav{5159} and \wav{5262}.}
  \label{fig:fe-ii-lines}
\end{figure*}
\begin{figure*}
  \centering
  \TwoPVb{Fe_II_5159}{Fe_II_5262}{0.75}{line}
  \caption{Continuum fluorescence-excited forbidden lines of singly-ionized iron: [\ion{Fe}{2}] \wav{5159} and \wav{5262}.}
  \label{fig:fe-ii-lines}
\end{figure*}

\subsubsection{\ion{Si}{2} lines}
\label{sec:silicon}

\begin{figure*}
  \centering
  \TwoPV{Si_II_6347}{Si_II_6371}{0.75}{line}
  \caption{Continuum fluorescence/recombination-excited permitted lines of singly-ionized silicon: \ion{Si}{2} \wav{6347} and \wav{6371}.}
  \label{fig:si-ii-lines}
\end{figure*}
\begin{figure*}
  \centering
  \TwoPVb{Si_II_6347}{Si_II_6371}{0.75}{line}
  \caption{Continuum fluorescence/recombination-excited permitted lines of singly-ionized silicon: \ion{Si}{2} \wav{6347} and \wav{6371}.}
  \label{fig:si-ii-lines}
\end{figure*}

\section*{Acknowledgements}

WJH and NFF acknowledge financial support from DGAPA-UNAM through project PAPIIT IN102012 and from a postdoctoral fellowship to NFF\@. 



\bibliographystyle{mn2e}
\bibliography{BibdeskLibrary}




\end{document}
